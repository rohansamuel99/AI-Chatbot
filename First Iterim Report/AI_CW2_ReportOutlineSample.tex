\documentclass[11pt]{cmpreport}

\usepackage[]{graphics}
\usepackage{natbib}
\usepackage[margin=2cm]{geometry}

\usepackage{rotating}
\usepackage{subfloat}
\usepackage{color}

%opening
\title{Developing an Intelligent Chatbot: the First Interim Report}
\author{Group 26: Adam Biggs 1*, Samuel Bedeau 2 and Rohan Samuel 3}


\begin{document}

\maketitle

\begin{abstract}
	This interim report presents (1) the outline of our coursework report, (2) some initial descriptions of the requirements of the coursework, the methods, programming languages, packages, tools that have been identified so far, and (3) an initial work plan.      

\end{abstract}

\section{Introduction}
A chatbot is a software application designed to converse with a human via text messages, in an as convincingly human way as possible. Chatboxes are often used on websites to automate the provision of information from specific human definable questions. Using artificial intelligence and natural language processing, a chatbot can simulate conversation with a user to help them complete a task, or answer a question they may have.

Intelligent chatbox systems are useful for a wide range of reasons. Most often they are useful for automating tasks which would otherwise have required a human to operate. They equally can be built to to provide information on a wide range of topics.

\subsection{Background and Motivation}
%A bit background information on chatbot in general and the coursework specification\citep{AI2018CW}.

The program we are required to create is a chatbox system for finding the cheapest available train ticket for a chosen journey. The system is also required to use Artificial Intelligence techniques to improve customer service satisfaction. 

\subsection{Aim and Objectives of this coursework} 
%You may rephrase the the aim and objectives from your point of view. 

The aim of the coursework task is to create an intelligent conversation system as a small group project. The task requires the use of complex Artificial Inteligence techniques to provide the users with the most natural human interaction possible.

The project has a number objectives which we are comparing using the use of the MoSCoW analysis, to determine the feasibility and priority of each goal.

\subsubsection{Must}
\begin{itemize}
\item The system will process data provided in a database format.
\item test
\end{itemize}
\subsubsection{Should}
\begin{itemize}
\item test
\end{itemize}
\subsubsection{Could}
\begin{itemize}
\item test
\end{itemize}
\subsubsection{Will Not}
\begin{itemize}
\item test
\end{itemize}
\subsection{Difficulties and Risks}
%List as many as you can identify. 
The risks of the project stem predominately from the objectives identified above. In addition there are also a few risks noted below: 
\begin{itemize}
    \item Complexity of the task and topics 
    \item Failing to understand what the user asked, failure in understanding sentiment
    \item Providing meaningful responses
\end{itemize}

\subsection{Work Plan}
%Draw a Gantt Chart and put it here.


\begin{cmpfigure}{Project Gantt chart \label{pplan}}
\begin{sideways}
\begin{ganttchart}[x unit=0.45cm, vgrid, title label font=\scriptsize,
canvas/.style={draw=black, dotted},
/pgfgantt/milestone left shift = 0,
/pgfgantt/milestone right shift = 0
]{1}{28}
\gantttitle{Project schedule shown for e-vision week numbers
 and semester week numbers}{28} \\
\gantttitlelist{1,...,28}{1}\\

   \ganttbar{Project proposal}{1}{3}                 %elem0  
\\ \ganttbar{Literature review}{3}{5}                %elem1 
\\ \ganttbar{Design}{6}{8}                           %elem2
\\ \ganttbar{Must (Coding)}{9}{13}                   %elem3
\\ \ganttmilestone{Minimal Viable Product}{14}       %elem4
\\ \ganttbar{Should (Coding)}{14}{17}                %elem5
\\ \ganttbar{Could (Coding)}{18}{19}                 %elem6
\\ \ganttbar{Testing}{9}{19}                         %elem7
\\ \ganttmilestone{Code delivery}{20}                %elem8
\\ \ganttbar{Final report writing}{21}{25}           %elem9
\\ \ganttmilestone{Portfolio submission}{26}         %elem10
\\ \ganttbar{Inspection preparation}{27}{28}         %elem11

\ganttlink{elem0}{elem1} 
\ganttlink{elem1}{elem2} 
\ganttlink{elem2}{elem3} 
\ganttlink{elem2}{elem7} 
\ganttlink{elem3}{elem4} 
\ganttlink{elem4}{elem5}
\ganttlink{elem5}{elem6}
\ganttlink{elem6}{elem8}
\ganttlink{elem7}{elem8}
\ganttlink{elem8}{elem9}

\ganttlink{elem9}{elem10}
\ganttlink{elem10}{elem11}


\end{ganttchart}
\end{sideways}
\end{cmpfigure}

\section{Related Work} 
%Review some similar chatbot systems. (Write as much as you have now.)  

https://www.virginmedia.com/

https://www.dpd.co.uk/content/how-can-we-help/contact.jsp

https://www.santander.co.uk/personal/support

https://help.disneyplus.com/csp

\section{Methods, Tools and Frameworks}
The methods we plan to use is a combination of the following techniques: 
\begin{itemize}
    \item Prediction Models: K-nearest Neighbours, bayesian model, Markov, Neural Networks
    \item Reasoning Engine: Infer answers from user
    \item Knowledge base: Represented by Questions and Answers, rules
    \item NLPU: Process and understand questions from user
    \item Database: Store previous conversations, data about subject area
\end{itemize}
We will be using Python as a programming language, libraries such as spaCy, nltk, and, also research several other libraries that will be of help to us in our development.

\subsection{Methods}

We will prioritise primarily on having a command line interface to interact with our system and if we have more time we will develop a graphical user interface.
In addition to this we will look into NLP methods such as tokenisation, and supervised learning methods such as neural networks to process the text that is inputted by the user. For prediction methods, we will look into libraries such as SciKit-Learn, and in terms of algorithms: K-Nearest Neighbours.
\subsection{Languages, Packages, Tools}

On programming languages: We will be using Python. 
Packages: for NLP, we will use NLTK\citep{NLTK}, for a start and branch into others when researching. 
For KnowledgeBase and Engine: PyKE or PyKnow. 
For Database: We will use Postgres.
Some of these may be subject to change if we find a package/integration more suited to us and shall be updated accordingly in our documentation.
 
\subsection{Development Framework}


\section{Design of the Chatbot}

 
\subsection{The Architecture of the chatbot}
You may draw a functional diagram if you like.  

You can describe your design for each key module or component of your chatbot, in a subsection. E.g. 
\subsection{User Interface} 

\subsection{NLP}

\subsection{Knowledgebase}

\subsection{Inferring Engine}

\subsection{Delay Prediction Models}

\subsection{Conversation Control}

%\begin{table}
%\centering
%\caption{This table lists ......}
%
%\begin{tabular}{|c|c|c|c|c|c|}
%\hline Methods &  &  &  &  &  \\ 
%\hline  &  &  &  &  &  \\ 
%\hline  &  &  &  &  &  \\ 
%\hline 
%\end{tabular} 
%\label{TableCC}
%\end{table}

\section{Implementation}

\section{Testing}

\subsection{Unit Testing}

\subsection{Integration Testing}

\subsection{System Testing}

\subsection{Userbility Testing}


\section{Evaluation and Discussion}

\section{Conclusion or Summary}

\bibliographystyle{agsm}
%\bibliographystyle{apalike}
% you should use your own bibtex file to replace the following example_ref bib file.
\bibliography{example_refs} 

\end{document}
